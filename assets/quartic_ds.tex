%! TEX root = ../../main.tex
%% Illustration of step-sizes bound from Armijo line-search. 

\begin{tikzpicture}[
        declare function={
                objective(\x)=      (\x<=0) * (pow(\x, 2)*0.8 - 1*\x)
                + (\x>0) * (2.0*pow(\x, 3) - 2 * \x^2 - 1*\x);
                objectivePrime(\x)=      (\x<=0) * (2*\x*0.8 - 1)
                + (\x>0) * (6.0*pow(\x, 2) - 4 * \x - 1);
                upperBound(\x)=     ( 1.75*pow(\x, 2) - 1*\x);
                M(\x) = 1.2*abs(objectivePrime(\x) + 1) / abs(\x);
                directionalSmoothness(\x) = -1 * \x
                + M(\x) * pow(\x, 2) / 2;
            }
    ]
    \begin{axis}[
            width=0.5\textwidth,
            height=6cm,
            axis x line=none, axis y line=none,
            ymin=-1.5, ymax=7, ytick={-5,...,7}, ylabel=$y$,
            xmin=-2.5, xmax=5, xtick={-5,...,5}, xlabel=$x$,
            restrict y to domain=-3.25:10,
            scale=3.5,
        ]

        \addplot[name path=function, domain=-3.5:5.5, samples=200, objective, line width=8pt]{objective(x)};
        \addplot[name path=directionalSmoothness, domain=-3.5:5.5, samples=300, bound, line width=8pt]{directionalSmoothness(x)};
        %\addplot[name path=upperBound, domain=-3.5:5.5, samples=200, oracle, line width=8pt]{upperBound(x)};

        \node[label={[font=\huge]270:$\xk$},circle,fill,inner sep=6pt] at (axis cs:0,0) {};

        \node[label={[font=\huge]180:$f(w)$}] at (axis cs:3,1) {};
        \node[label={[font=\Large]180:$M(\xk, y)$-Smooth}] at (axis cs:1,6) {};
        \node[label={[font=\Large]180:Bound}] at (axis cs:0.1,5.2) {};
        %\node[label={0:{$\ell_{v_k}(\eta)$}}] at (axis cs:3.6,-2.4) {};
        %\node[label={180:{$h_{v_k}(\eta)$}}] at (axis cs:0.3,3) {};

        \addplot fill between[
                of = function and directionalSmoothness,
                split, % calculate segments
                every odd segment/.style = {fill=green, fill opacity=0.3},
                every even segment/.style  = {fill=green, fill opacity=0.3}
            ];

    \end{axis}

\end{tikzpicture}
